% ========================================
%	Header einbinden
% ========================================

\input{longheader.tex}
\usepackage{multirow}
\usepackage{color}

% ========================================
%	Angaben für das Titelblatt
% ========================================

\title{Versuch 28 - Elektronenspinresonanz\\% Titel des Versuchs 
\large TU Dortmund, Fakultät Physik\\ 
\normalsize Fortgeschrittenen-Praktikum}

\author{Jan Adam\\			% Name Praktikumspartner A
{\small \href{jan.adam@tu-dortmund.de}{jan.adam@tu-dortmund.de}}	% Erzeugt interaktiven einen Link
\and						% um einen weiteren Author hinzuzfügen
Dimitrios Skodras\\					% Name Praktikumspartner B
{\small \href{dimitrios.skodras@tu-dortmund.de}{dimitrios.skodras@tu-dortmund.de}}		% Erzeugt interaktiven einen Link
}
\date{30. April 2014}				% Das Datum der Versuchsdurchführung

% ========================================
%	Das Dokument beginnt
% ========================================

\begin{document}

% ========================================
%	Titelblatt erzeugen
% ========================================

\maketitle					% Jetzt wird die Titelseite erzeugt
\thispagestyle{empty} 				% Weder Kopfzeile noch Fußzeile

% ========================================
%	Der Vorspann
% ========================================

%\newpage					% Wenn Verzeichnisse auf einer neuen Seite beginnen sollen
%\pagestyle{empty}				% Weder Kopf- noch Fußzeile für Verzeichnisse

\tableofcontents

%\newpage					% eine neue Seite
%\thispagestyle{empty}				% Weder Kopf- noch Fußzeile für Verzeichnisse
%\listoffigures					% Abbildungsverzeichnis

%\newpage					% eine neue Seite
%\thispagestyle{empty}				% Weder Kopf- noch Fußzeile für Verzeichnisse
%\listoftables					% Tabellenverzeichnis
\newpage					% eine neue Seite


% ========================================
%	Kapitel
% ========================================

%\section{Einleitung}				% Bei Bedarf

\section{Theorie}
\setcounter{page}{1}
Die Hüllenelektronen eines Atoms, die einen Bahndrehimpuls besitzen, erzeugen auch
ein magnetisches Moment. Laut einiger Experimente haben Elektronen jedoch außerdem einen "Eigendrehimpuls" genannt "Spin", der experimentell zu Tage tritt, wenn der Gesamtdrehimpuls der Elektronenhülle verschwindet. Dies ist klassisch nicht zu erklären, da Elektronen punktförmig sind und keine Ausdehnung haben.\\
In diesem Experiment soll nun versucht werden, den Betrag dieses Drehimpulses zu bestimmen.\\

Mit Überlegungen aus der Quantenmechanik lässt sich das gesuchte Bohrsche Magneton herleiten: Die Stromdichte eines Teilchenstroms aus Elektronen ist 
\begin{align}
\vec{S} = \frac{\hbar}{2im_0} (\Psi^* \nabla \Psi - \Psi \nabla \Psi^*)
\end{align}
Mit $\hbar$ dem Planckschen Wirkungsquantum und $m_0$ der Elektronenmasse.\\

Setzt man für $\Psi$ die Kugelförmige Wellenfunktion des Elektrons ein und integriert über die Querschnittsfläche der Elektronenhülle, so erhält man das Bohrsche Magneton $\mu_B$:
\begin{align}
\mu_B := -\frac{1}{2}\frac{e_0}{m_0}\hbar= (9,274015 \pm 0,000003)\cdot10^{-24} \frac{J}{T}
\end{align}
Neben dem Betrag des Drehimpulses ist auch seine Richtung gequantelt. Bezüglicher einer ausgezeichneten Raumachse kann die Komponente $l_z$ nur ganzzahlige Vielfache von $\hbar$ annehmen:
\begin{align}
l_z=m_l\hbar
\end{align}
Mit $m_l = 0, \pm 1, \pm 2,\cdots , \pm l$\\

Wird dieses magnetische Moment in ein äußeres Magnetfeld eingeführt, so erhält es, abhängig von seiner Ausrichtung, potentielle Energie. Seine Grundenergie $E_0$ spaltet in $2l+1$ Unterniveaus auf, der Zeeman-Effekt wird beobachtet.\\

Das Magnetische Moment wird nun zweckmäßigerweise in Einheiten des Bohrschen Magnetons ausgedrückt der dabei auftretende Proportionalitätsfaktor wird \glqq Landé-Faktor\grqq $g$  genannt. Dieser soll im Folgenden berechnet werden.
\section{Durchführung und Aufbau}
Zur Bestimmung des g-Faktors wird die Elektronenspin-Resonanz-Methode (ESR) benutzt. Dazu wird eine Substanz mit freien Elektronen in ein homogenes Magnetfeld eingeführt und das Energieniveau $E_0$ mittels Zeeman-Effekt aufgespalten.\\

\begin{figure}[H]
\includegraphics[scale=0.5]{../pics/aufsp.pdf}
\caption{Darstellung der Aufspaltung des Grundniveaus in zwei weitere Niveaus für $B\neq 0$}
\label{pic_aufspaltung}
\end{figure}

Gemäß der Maxwell-Boltzmann-Verteilung ist der obere Zustand schwächer besetzt als der untere. Durch Einstrahlung von Lichtquanten mit der Differenzenergie von etwa $\hbar\nu \approx 6\cdot10^{-7}eV \Rightarrow \nu \approx 140 \text{MHz}$ kann man die Elektronen in das höhere Niveau anheben, wobei sie ihren Spin umklappen. Dies ist die sog. "Elektronenspin-Resonanz".\\

Das Hochfrequenzfeld wird durch eine kleine Spule induziert. Das homogene Magnetfeld erzeugt eine Helmholtzspule. Wird die Resonanzfrequenz erreicht, so ändert sich die makroskopische Magnetisierung der Probe und diese wird an der angeschlossenen Brückenspannung sichtbar, da diese nun aus ihrer abgeglichenen Einstellung verstimmt wird.

In Abbildung \ref{pic_aufbau} ist der Aufbau der Messapparatur zu erkennen.
\begin{figure}[H]
\includegraphics[scale=0.4]{../pics/aufbau.pdf}
\caption{Aufbau der im Versuch verwendeten Messapparatur}
\label{pic_aufbau}
\end{figure}

Die Brückenspannung wird nun verstärkt, gefiltert und zusammen mit dem Output eines Rampengenerators für die Spulenspannung an einen X-Y-Schreiber angeschlossen. Im Idealfall sollte eine Resonanz in Form von Abbildung \ref{pic_resonanz} sichtbar werden. Aus dieser kann dann g berechnet werden.
\begin{figure}
\includegraphics[scale=0.5]{../pics/resonanz.pdf}
\caption{Absorbtions-Resonanzsignal}
\label{pic_resonanz}
\end{figure}


\section{Auswertung}
\subsection{Kalibrierung der Messachsen}
\label{sec_Kalib}
Zur Bestimmung der für die Auswertung relevanten Resonanzstellen, ist es erforderlich, die im Anhang aufgeführten Messkurven entlang der X-Achse
nach den gemessenen Spannungen $U_\text{Rampe}$ = $U_\text{R}$ zu kalibrieren. 
\begin{equation}
 U_\text{R} = m\cdot x + U_0
\end{equation}

Hierzu dienen die Werte aus den Tabellen \ref{tab_xAxisKalib15MHz} bis
\ref{tab_xAxisKalib30MHz}. Rot geschriebene Zahlen zeichnen hier die Resonanzstelle
\renewcommand{\arraystretch}{1.2}
\begin{table}[H]
 \begin{tabular}{c|c|c|c||c|c|c|c}
  $x$ in cm & $y$ in cm & $U_\text{R}$ in V & $U_\text{B} $ in mV & $x$ in cm & $y$ in cm & $U_\text{R}$ in V & $U_\text{B}$ in mV \\
 \hline
 0,0&	0,0&	10,0&	181,0&	0,0&	0,0&		5,0&	173,0\\
4,0&	0,0&	14,1&	181,0&	4,0&	0,0&		9,8&	173,0\\
7,0&	0,3&	17,2&	183,5&	7,0&	0,2&		13,5&	174,0\\
8,0&	0,5&	18,2&	185,2&	9,0&	0,5&		15,9&	175,5\\
9,0&	0,6&	\textcolor{red}{19,2}&	\textcolor{red}{186,0}&	10,0&	0,6&		\textcolor{red}{17,1}&	\textcolor{red}{176,0}\\
10,0&	0,6&	\textcolor{red}{20,2}&	\textcolor{red}{186,0}&	11,0&	0,5&		18,3&	175,5\\
11,0&	0,5&	21,3&	185,2&	12,0&	0,4&		19,5&	175,0\\
12,0&	0,4&	22,3&	184,3&	13,0&	0,2&		20,8&	174,0\\
14,0&	0,1&	24,3&	181,8&	16,0&	0,0&		24,4&	173,0\\
17,0&	0,0&	27,4&	181,0&	19,8&	0,0&		29,0&	173,0\\ 
21,5&	-0,1&	32,0&	180,2&	&	&		&	\\

 \end{tabular}
 \caption{Messkurven bei $\nu_e$ = 14,798 kHz; links $B_\text{parallel}$ und rechts $B_\text{parallel}$}
 \label{tab_xAxisKalib15MHz}
\end{table}
\begin{table}[H]
 \begin{tabular}{c|c|c|c||c|c|c|c}
  $x$ in cm & $y$ in cm & $U_\text{R}$ in V & $U_\text{B} $ in mV & $x$ in cm & $y$ in cm & $U_\text{R}$ in V & $U_\text{B}$ in mV \\
 \hline
0,0	&0,0&	15,0&	182,8	&0,0&	0&	14,0&	156,0\\
2,0&	0,2&	16,8&	183,9&	3,0&	0&	17,5&	156,0\\
4,0&	0,3&	18,7&	184,5&	6,0&	0,3&	20,9&	158,1\\
6,0&	0,9&	20,5&	187,8&	7,0&	0,5&	22,1&	159,6\\
8,0&	2,3&	22,4&	195,5&	8,0&	0,7&	\textcolor{red}{23,2}&	\textcolor{red}{161,0}\\
9,0&	3,1&	23,3&	199,9&	9,0&	0,6&	24,4&	160,3\\
10,0&	3,9&	24,2&	204,4&	10,0&	0,6&	25,5&	160,3\\
10,5&	4,6&	24,7&	208,2&	12,0&	0,2&	27,8&	157,4\\
11,0&	4,9&	25,1&	209,9&	16,5&	0,1&	33,0&	156,7\\
11,5&	5,1&	\textcolor{red}{25,6}&	\textcolor{red}{211,0}&		&&		&\\
12,0&	5,1&	\textcolor{red}{26,1}&	\textcolor{red}{211,0}&		&&		&\\
12,5&	4,9&	26,5&	209,9&		&&		&\\
13,0&	4,6&	27,0&	208,2&		&&		&\\
13,5&	4,3&	27,4&	206,6&		&&		&\\
14,0&	3,9&	27,9&	204,4&		&&		&\\
15,0&	3,0&	28,8&	199,4&		&&		&\\
16,0&	2,6&	29,7&	197,2&		&&		&\\
17,0&	2,1&	30,7&	194,4&		&&		&\\
19,0&	1,4&	32,5&	190,6&		&&		& \\

 \end{tabular}
 \caption{Messkurven bei $\nu_e$ = 19,448 kHz; links $B_\text{parallel}$ und rechts $B_\text{parallel}$}
 \label{tab_xAxisKalib20MHz}
\end{table}
\begin{table}[H]
 \begin{tabular}{c|c|c|c||c|c|c|c}
  $x$ in cm & $y$ in cm & $U_\text{R}$ in V & $U_\text{B} $ in mV & $x$ in cm & $y$ in cm & $U_\text{R}$ in V & $U_\text{B}$ in mV \\
 \hline
0,0 &	0,0&	23,0&	177,0&	0,0&	0,7&	17,0&	211,0\\
2,0&	0,1&	25,1&	177,8&	1,0&	0,0&	18,3&	205,0\\
4,0&	0,7&	27,1&	182,4&	2,0&	0,0&	19,6&	205,0\\
5,0&	0,9&	28,1&	184,0&	3,0&	0,3&	20,9&	207,6\\
6,0&	1,4&	29,2&	187,9&	4,0&	0,9&	22,2&	212,7\\
6,5&	1,6&	29,7&	189,4&	5,0&	1,6&	23,5&	218,7\\
7,0&	1,7&	30,2&	190,2&	6,0&	2,8&	24,9&	229,0\\
7,5&	1,8&	\textcolor{red}{30,7}&	\textcolor{red}{191,0}&	7,0&	4,5&	26,2&	243,6\\
8,0&	1,8&	\textcolor{red}{31,2}&	\textcolor{red}{191,0}&	7,5&	5,4&	26,8&	251,3\\
9,0&	1,4&	32,3&	187,9&	8,0&	6,3&	27,5&	259,0\\
10,0&	1,0&	33,3&	184,8&	8,5&	6,9&	28,1&	264,1\\
11,0&	0,8&	34,3&	183,2&	9,0&	7,0&	\textcolor{red}{28,8}&	\textcolor{red}{265,0}\\
13,0&	0,3&	36,4&	179,3&	9,5&	6,9&	29,4&	264,1\\
15,0&	0,1&	38,4&	177,8&	10,0&	6,5&	30,1&	260,7\\
17,5&	-0,1&	41,0&	176,2&	10,5&	5,6&	30,8&	253,0\\
&	&	&	&	11,0&	4,6&	31,4&	244,4\\
&	&	&	&	11,5&	3,8&	32,1&	237,6\\
&	&	&	&	13,0&	2,5&	34,0&	226,4\\
&	&	&	&	14,0&	1,4&	35,3&	217,0\\
&	&	&	&	14,5&	1,1&	36,0&	214,4\\
&	&	&	&	16,8&	1,1&	39,0&	214,4 \\

 \end{tabular}
 \caption{Messkurven bei $\nu_e$ = 23,888 kHz; links $B_\text{parallel}$ und rechts $B_\text{parallel}$}
 \label{tab_xAxisKalib24MHz}
\end{table}
\begin{table}[H]
 \begin{tabular}{c|c|c|c||c|c|c|c}
  $x$ in cm & $y$ in cm & $U_\text{R}$ in V & $U_\text{B} $ in mV & $x$ in cm & $y$ in cm & $U_\text{R}$ in V & $U_\text{B}$ in mV \\
 \hline
0,0 &	0,0&	24,0&	222,0&	0,0&	0,1&	26,0&	181,7\\
2,0&	0,1&	26,1&	222,7&	1,0&	0,0&	27,1&	181,0\\
4,0&	0,2&	28,3&	223,4&	1,5&	0,1&	27,6&	181,7\\
6,0&	0,5&	30,4&	225,4&	2,0&	0,3&	28,2&	183,1\\
7,0&	0,7&	31,5&	226,8&	4,0&	0,6&	30,3&	185,1\\
8,0&	1,1&	32,6&	229,5&	5,0&	0,5&	31,4&	184,4\\
8,5&	1,5&	33,1&	232,2&	6,0&	0,8&	32,5&	186,5\\
9,0&	2,2&	33,6&	237,0&	7,0&	1,2&	33,5&	189,3\\
10,0&	2,8&	34,7&	241,1&	8,0&	1,5&	34,6&	191,3\\
10,5&	3,3&	35,3&	244,5&	8,5&	1,6&	\textcolor{red}{35,2}&	\textcolor{red}{192,0}\\
11,0&	3,6&	35,8&	246,6&	9,0&	1,6&	\textcolor{red}{35,7}&	\textcolor{red}{192,0}\\
12,0&	4,0&	36,9&	249,3&	9,5&	1,5&	36,2&	191,3\\
12,5&	4,1&	\textcolor{red}{37,4}&	\textcolor{red}{250,0}&	11,5&	0,8&	38,4&	186,5\\
13,0&	4,0&	37,9&	249,3&	12,5&	0,8&	39,5&	186,5\\
14,0&	3,5&	39,0&	245,9&	15,0&	0,4&	42,2&	183,8\\
15,0&	2,8&	40,1&	241,1&	17,0&	0,3&	44,3&	183,1\\
15,5&	2,1&	40,6&	236,3&	19,5&	0,2&	47,0&	182,4\\
16,0&	1,7&	41,1&	233,6&	&	&	&	\\
17,0&	1,3&	42,2&	230,9&	&	&	&	\\
19,0&	0,7&	44,4&	226,8&	&	&	&	\\
19,6&	0,6&	45,0&	226,1&	&	&	&	\\

 \end{tabular}
 \caption{Messkurven bei $\nu_e$ = 29,448 kHz; links $B_\text{parallel}$ und rechts $B_\text{parallel}$}
 \label{tab_xAxisKalib30MHz}
\end{table}
Hieraus ergeben sich nun die jeweiligen Kalibrierungsfaktoren in Tabelle \ref{tab_xAxisKalibTOTAL}. Mit einem Widerstand von 50 $\Omega$ lassen sich
die Rampengeneratorspannungen in Ströme $I$ umrechnen.
\begin{table}[H]
 \begin{tabular}{c|c|c|c}
  $\nu_e$ in MHz & $m$ in V/cm & $U_\text{res}$ in V & $I_\text{res}$ in mA\\
  \hline
  14,798 & 1,02 & 19,72 & 394,4 \\
  & 1,21&17,12 & 342,4\\
  19,448&0,92 &25,82 & 516,4\\
  &1,15&23,21& 464,2\\
  23,888&1,03&30,97& 619,4\\
  &1,31&28,79&575,7\\
  29,448&1,07&37,39&747,8\\
  &1,08&35,42&708,5\\
 \end{tabular}
\caption{Zusammengefasste Werte}
\label{tab_xAxisKalibTOTAL}
\end{table}

\subsection{Magnetische Flussdichte der Erde}
Mit den Ergebnissen aus Abschnitt \ref{sec_Kalib} lässt sich nun das Erdmagnetfeld bestimmen. Hierzu werden die Ströme $I_\text{res}$ aus Tabelle
\ref{tab_xAxisKalibTOTAL} nach der Gleichung für die Helmholtzspulen 
\begin{align}
B(I)=\frac{8}{\sqrt{125}} \mu_0 \frac{n}{r}I
\end{align}
in magnetische Flussdichten umgerechnet, wobei der Radius
$r$ = 0,1 m und die Windungszahl N = 156 ist. Da für gleiche Frequenzen $\nu_e$ die Resonanzstellen verschieden sind, bewirkt das Erdmagnetfeld einen
Einfluss in die jeweilige Richtung, je nachdem, ob die Spule parallel oder antiparallel dazu ausgerichtet ist. Diesen Einfluss kann man ermitteln,
indem die zueinander gehörenden Flussdichten voneinander abgezogen und das Ergebnis halbiert wird
\begin{equation}
 B_\text{Erde} = \frac12 ( B_\text{par} - B_\text{antipar}).
\end{equation}
In Tabelle \ref{tab_erdMagnetfeld} sind die jeweiligen Magnetfelder und das errechnete Erdmagnetfeld aufgeführt,
\begin{table}[H]
 \begin{tabular}{c|c|c}
 $B_\text{par}$ in $\mu$T & $B_\text{antipar}$ in $\mu$T& $B_\text{Erde}$ in $\mu$T\\
 \hline
 553,3 &	480,3&	36,5\\
724,4&	651,2&	36,6 \\
868,9&	807,6&	30,7\\
1049,0&	993,8&	27,6\\
 \end{tabular}
\caption{magnetische Flussdichte der Erde}
\label{tab_erdMagnetfeld}
\end{table}
was zu einem Erdmagnetfeld führt von 
\begin{equation}
 B_\text{Erde} = 32,8 \pm 1,9 \mu\text{T}.
\end{equation}

\renewcommand{\arraystretch}{1.0}
\subsection{gyromagnetisches Verhältnis des Elektrons}
Mit den vom Erdmagnetfeld bereinigten Spulenfeldern lässt sich nun schließlich das gyromagnetische Verhältnis des Elektrons nach \eqref{eq_nu(B).18}
berechnen. Hierzu wird eine Fitgerade 
\begin{equation}
 B = \frac{h}{\mu_B \cdot g} \nu = m_\text{gyro} \cdot \nu
\end{equation}
mit GNUplot durch die Wertepaare \{$\nu_{e,i},B_{\text{res},i}$\} gelegt, was in Abbildung \ref{pic_gyro} zu sehen ist
\begin{figure}[H]
\includegraphics[width=0.8\textwidth]{../pics/landefaktor.pdf}
\caption{Gyromagnetisches Verhältnis oder Landéfaktor des Elektrons}
\label{pic_gyro}
\end{figure}
und zu einem Steigungsparameter 
\begin{equation}
m_\text{gyro} = 0,0343 \pm 0,0006 \text{ Ts}
\end{equation}
führt und damit schließlich den Landéfaktor berechnen lässt.
\begin{equation}
 g = \frac{h}{\mu_B\cdot m_\text{gyro}} = 2,082 \pm 0,037
\end{equation}


\section{Diskussion}
\subsection{Erdmagnetfeld}
Der berechnete Wert für das Erdmagnetfeld steht in folgendem Verhältnis zum Literaturwert \cite{ErdB}
\begin{equation}
 \frac{B_\text{Mess}}{B_\text{Lit}} = 68,3 \%.
\end{equation}
Die etwas hohe Abweichung spricht für eine unpräzise Ausrichtung der Helmholtzspule, was auf die sehr empfindliche Bussole zurückführbar ist, die
auf diverse andere elektrische Geräte reagiert.

\subsection{Landéfaktor}
Das gyromagnetische Verhältnis wurde durch einen Fit ermittelt und hat eine Übereinstimmung zum Literaturwert \cite{Gyro} von
\begin{equation}
 \frac{g_\text{Mess}}{g_\text{Lit}} = 103,4 \%.
\end{equation}
Unabhängig von den Abweichungen vom Erdmagnetfeld ist eine sehr gute Bestimmung des Landéfaktors gelungen. 




\begin{thebibliography}{xxxxxxxxxxxxxxxxx}
 \bibitem[Chemie.de]{ErdB}Form und Stärke des Erdmagnetfelds\\ \href{http://www.chemie.de/lexikon/Erdmagnetfeld.html#Form_und_St.C3.A4rke_des_Erdmagnetfeldes}{chemie.de/lexikon/Erdmagnetfeld}
 \bibitem[Universal Lexikon]{Gyro}Landé-Faktor\\ \href{http://universal\_lexikon.deacademic.com/144745/}{universal\_lexikon.deacademic.com/144745/}
 \bibitem[Versuchsanleitung]{}{Versuch V28 Elektronen-Spin-Resonanz }
\end{thebibliography}

% ========================================
%	Literaturverzeichnis
% ========================================

%\bibliographystyle{plainnat}			% Bibliographie-Style auswählen
%\bibliography{BIBDATEI}			% Literaturverzeichnis

% ========================================
%	Das Dokument endent
% ========================================

\end{document}
