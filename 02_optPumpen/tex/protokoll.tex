% ========================================
%	Header einbinden
% ========================================

\input{longheader.tex}

% ========================================
%	Angaben für das Titelblatt
% ========================================

\title{Versuch 21 - Optisches Pumpen\\				% Titel des Versuchs 
\large TU Dortmund, Fakultät Physik\\ 
\normalsize Fortgeschrittenen-Praktikum}

\author{Jan Adam\\			% Name Praktikumspartner A
{\small \href{jan.adam@tu-dortmund.de}{jan.adam@tu-dortmund.de}}	% Erzeugt interaktiven einen Link
\and						% um einen weiteren Author hinzuzfügen
Dimitrios Skodras\\					% Name Praktikumspartner B
{\small \href{dimitrios.skodras@tu-dortmund.de}{dimitrios.skodras@tu-dortmund.de}}		% Erzeugt interaktiven einen Link
}
\date{05. März 2014}				% Das Datum der Versuchsdurchführung

% ========================================
%	Das Dokument beginnt
% ========================================

\begin{document}

% ========================================
%	Titelblatt erzeugen
% ========================================

\maketitle					% Jetzt wird die Titelseite erzeugt
\thispagestyle{empty} 				% Weder Kopfzeile noch Fußzeile

% ========================================
%	Der Vorspann
% ========================================

%\newpage					% Wenn Verzeichnisse auf einer neuen Seite beginnen sollen
%\pagestyle{empty}				% Weder Kopf- noch Fußzeile für Verzeichnisse

\tableofcontents

%\newpage					% eine neue Seite
%\thispagestyle{empty}				% Weder Kopf- noch Fußzeile für Verzeichnisse
%\listoffigures					% Abbildungsverzeichnis

%\newpage					% eine neue Seite
%\thispagestyle{empty}				% Weder Kopf- noch Fußzeile für Verzeichnisse
%\listoftables					% Tabellenverzeichnis
\newpage					% eine neue Seite


% ========================================
%	Kapitel
% ========================================

%\section{Einleitung}				% Bei Bedarf

\section{Theorie}
\setcounter{page}{1}

\section{Durchführung}

\section{Auswertung}
\subsection{Vertikalkomponente des Erdmagnetfelds}
Um dem Einfluss des vertikalen Magnetfelds entgegenzuwirken, wird eine Helmholtz-Spule in die entsprechende Richtung orientiert, sodass das vom 
Stromfluss $I$ induzierte Magnetfeld $B$ genau dem des Erdmagnetfelds entspricht. $N$=20 ist dabei die Windungszahl und $r$=11,735 cm der Radius
der Spule. $\mu_0$ ist die Vakuumpermeabilität. Bei einer Spannung von $U$=1,99 V und einem Widerstand von $R$ = 1 $\Omega$ wird kein Einfluss
des vertikalen Erdmagnetfelds mehr festgestellt. Die Feldstärke $B_{\text{vert,Erde}}$ lässt nach Biot-Savart berechnen mit
\begin{equation}
 B_{\text{vert,Erde}} = \mu_0 \frac{8}{\sqrt{125}}\frac{I N}{R} = 30,5 \, \mu\text{T}.
\end{equation}
\subsection{Horizontalkomponente des Erdmagnetfelds}



\parskip 340pt
\Large{Literatur}\\\\

% ========================================
%	Literaturverzeichnis
% ========================================

%\bibliographystyle{plainnat}			% Bibliographie-Style auswählen
%\bibliography{BIBDATEI}			% Literaturverzeichnis

% ========================================
%	Das Dokument endent
% ========================================

\end{document}
