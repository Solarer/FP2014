% ========================================
%	Header einbinden
% ========================================

\input{longheader.tex}

% ========================================
%	Angaben für das Titelblatt
% ========================================

\title{Versuch 21 - Optisches Pumpen\\				% Titel des Versuchs 
\large TU Dortmund, Fakultät Physik\\ 
\normalsize Fortgeschrittenen-Praktikum}

\author{Jan Adam\\			% Name Praktikumspartner A
{\small \href{jan.adam@tu-dortmund.de}{jan.adam@tu-dortmund.de}}	% Erzeugt interaktiven einen Link
\and						% um einen weiteren Author hinzuzfügen
Dimitrios Skodras\\					% Name Praktikumspartner B
{\small \href{dimitrios.skodras@tu-dortmund.de}{dimitrios.skodras@tu-dortmund.de}}		% Erzeugt interaktiven einen Link
}
\date{05. März 2014}				% Das Datum der Versuchsdurchführung

% ========================================
%	Das Dokument beginnt
% ========================================

\begin{document}

% ========================================
%	Titelblatt erzeugen
% ========================================

\maketitle					% Jetzt wird die Titelseite erzeugt
\thispagestyle{empty} 				% Weder Kopfzeile noch Fußzeile

% ========================================
%	Der Vorspann
% ========================================

%\newpage					% Wenn Verzeichnisse auf einer neuen Seite beginnen sollen
%\pagestyle{empty}				% Weder Kopf- noch Fußzeile für Verzeichnisse

\tableofcontents

%\newpage					% eine neue Seite
%\thispagestyle{empty}				% Weder Kopf- noch Fußzeile für Verzeichnisse
%\listoffigures					% Abbildungsverzeichnis

%\newpage					% eine neue Seite
%\thispagestyle{empty}				% Weder Kopf- noch Fußzeile für Verzeichnisse
%\listoftables					% Tabellenverzeichnis
\newpage					% eine neue Seite


% ========================================
%	Kapitel
% ========================================

%\section{Einleitung}				% Bei Bedarf

\section{Theorie}
\setcounter{page}{1}
Die Elektronenhüllen im Atom sind durch Energieniveaus von einander getrennt. Die äußeren Schalen haben dabei eine deutlich geringere Energiedifferenz, so dass zum Teil höhere Schalen allein durch thermische Fluktuationen besetzt werden. Die Besetzungszahlen $N_1$ und $N_2$ sind dabei durch die Boltzmannsche Gleichung
\begin{align}
\frac{N_2}{N_1}=\frac{g_2}{g_1}\frac{\text{exp}(-W_2/k_BT)}{\text{exp}(-W_1/k_BT)}
\label{eq_boltz}
\end{align}
miteinander verknüpft, solange sich die Atome bei der Temperatur T im thermischen Gleichgewicht befinden. Dabei sind die $g_i$ die sogenannten statistischen Gewichte, die angeben, wieviele Zustände zur Energie $W_i$ gehören.\\

Durch optisches Pumpen kann man nur eine Abweichung von der in Gleichung \eqref{eq_boltz} gegebenen Energieverteilung erreichen.\\
Elektronen können in ein höheres Niveau an oder abgeregt werden durch Absorbtion oder Emission eines Photons mit der Frequenz
\begin{align}
h\nu=W_2-W_1
\label{eq_quant} .
\end{align}

Für besonders kleine (dh: $h\nu << k_B T$) Übergänge, kann man diese Differenz mit hoher Präzision ausmessen. Dafür betrachtet man die Hyperfeinstrukturaufspaltung bzw. die Zeeman-Aufspaltung durch ein äußeres Magnetfeld. Durch weitere Rechnung kann man so zudem die Landeschen g-Faktoren und die Spins der Elektronenhülle und des Kerns berechnen.\\

\subsection{Zeeman-Effekt}
Aus dem Gesamtdrehimpuls $\vec{J}$ der Elektronenhülle des Atoms folgt ein magnetisches Moment
\begin{align*}
\vec{\mu_J}&=-g_J \mu_B \vec{J}\\
|\vec{\mu_J}|&=-g_J \mu_B \sqrt{J(J+1)}
\end{align*}
Hierin bedeuten $\mu_B$ das Bohrsche Magneton und $g_J$ den Landé-Faktor. Durch $g_J$ wird berücksichtigt, dass sich $\mu_J$ aus den magnetischen Momenten des Bahndrehimpulses L und des Spins S zusammensetzt:
\begin{align*}
\vec{\mu_L} &= -\mu_B \vec{L}\\
\vec{\mu_S} &= -g_S \mu_B \vec{S}
\end{align*}
oder
\begin{align*}
|\vec{\mu_L}|&=g_J \mu_B \sqrt{L(L+1)}\\
|\vec{\mu_S}|&=g_J \mu_B \sqrt{S(S+1)}
\end{align*}
mit $g_S = 2,00232$ dem Lande-Faktor des freien Elektrons. Das Gesamtmoment $\vec{\mu_J} = \vec{\mu_L}+\vec{\mu_S}$ führt nun eine Präzessionsbewegung um die $\vec{J}$-Richtung aus. Dabei mittelt sich die senkrechte Komponente von $\vec{\mu_B}$ heraus und wegen der Richtungsquantelung kann die Parallelkomponente nur ganzzahlige Vielfache $M_J$ annehmen:
\begin{align}
U_{mag}= M_Jg_J \mu_B B
\end{align}
diese Aufspaltung in $2J+1$ Unterniveaus bezeichnet man als Zeeman-Effekt.

\subsection{Hyperfeinstruktur}
Hat das Atom einen von Null verschiedenen Kernspin I, so koppeln I und J zum Gesamtdrehimpuls
\begin{align*}
\vec{F}=\vec{I}+\vec{J}
\end{align*}
und erzeugen eine weitere Aufspaltung der Energieniveaus, wie man in Grafik \ref{} sehen kann:
\begin{figure}[h]
\includegraphics[scale=1]{/pics/zeehyper.jpg}
\caption{Hyperfeinstruktur und Zeeman-Aufspaltung eines Atoms mit $I=3/2$}
\end{figure}
Auch hier findet sich wieder ein Lande-Faktor $g_F$:
\begin{align*}
|\vec{\mu_F}|&=g_F \mu_B \sqrt{F(F+1)}
\end{align*}
Diese Gleichung lässt sich aber noch umschreiben zu
\begin{align}
g_F\approx g_J \frac{F(F+1)+J(J+1)-I(I+1)}{2F(F+1)}
\end{align}

\subsection{Optisches Pumpen}
Ist das Atom im thermischen Gleichgewicht, so wird man keine Übergänge von Elektronen auf den Schalen erkennen können. Besetzt man die Schalen jedoch anders, so können Elektronen spontan oder durch Anregung ihre Schale wechseln. Dies erreicht man durch das Einstrahlen von Lichtquanten mit der passenden Energie (siehe Gleichung \eqref{eq_quant}). Da die Energieniveaus jedoch auch in den Spins aufspalten, muss nicht nur auf die Energie, sondern auch auf die richtige Polarisation des Lichtes geachtet werden, damit der Drehimpuls beim Übergang erhalten bleibt. Beispielhaft ist dies in Abbildung \ref{pic_uebergang} erkennbar:
\begin{figure}[h]
\includegraphics[scale=1]{./pics/uebergaenge.jpg}
\caption{Zeeman-Aufspaltung mit möglichen Übergängen und ihren Drehimpulsbeiträgen M}
\label{pic_uebergang}
\end{figure}
Es sind also mit $\Delta M_J =0,\pm 1$ insgesamt vier Strahlungsübergänge möglich, die alle verschiedene Energien und Polarisationen haben. Ein  $\sigma^+$-Übergang (mit  $M_J = +1$) entspricht rechtszirkular-polarisiertem Licht, das heißt, der Spin der Lichtquanten steht antiparallel zu ihrer Ausbreitungsrichtung; während beim $\sigma^-$-Übergang (  $M_J = -1$) Spin und Ausbreitungsrichtung parallel sind. Bei den $\pi$-Übergängen ($M_J$ = 0) wird
hingegen linear polarisiertes Licht emittiert und absorbiert. Die Polarisationsrichtung ist dabei parallel zu $\vec{B}$. Wegen des Dipolcharakters des Strahlungsfeldes erfolgt bei $\pi$-Übergängen keine Abstrahlung parallel zu $\vec{B}$; senkrecht zu $\vec{B}$ ist die Strahlungsintensität maximal. Dagegen beobachtet man das bei den $\sigma$-Übergängen emittierte zirkular-polarisierte Licht ausschließlich in Richtung von $\vec{B}$. In allen dazu senkrechten Richtungen erscheint es linear polarisiert.\\

Sind nun Alkali-Atome in einem Dampf im thermischen Gleichgewicht, so ist das $M_J=+\frac{1}{2}$ Niveau schwächer besetzt, als das energieärmere  $M_J=-\frac{1}{2}$ Niveau. Strahlt man nun rechtszirkular-polarisiertes $D_1$-Licht ein, so werden im Dampf Übergänge von $^2S_{\frac{1}{2}}M_J=-\frac{1}{2}$ nach $^2P_{\frac{1}{2}}M_J=+\frac{1}{2}$ induziert und der angeregte Zustand geht durch spontane Emission nach ca. $10^{-8}$s wieder in beide S-Grundzustände über. Wegen $\DeltaM_J=1$ werden aus dem $^2S_{\frac{1}{2}}M_J=+\frac{1}{2}$ Zustand jedoch keine Elektronen mehr durch Quanten angeregt, so dass mit der Zeit alle Elektronen der $^2S_{\frac{1}{2}}$-Zustände von $M_J=-\frac{1}{2}$ nach $M_J=\frac{1}{2}$ gepumpt werden. Je leerer das Niveau wird, desto weniger Quanten können absorbiert werden und desto durchsichtiger wird der Stoff für Licht der entsprechenden Wellenlänge und Polarisation.\\

Zwar kann theoretisch auch durch spontane Emission das untere Niveau wieder gefüllt werden, allerdings zeigt die Wahrscheinlichkeit für spontane Emission eine-$\nu^3$-Abhängigkeit, so dass für Übergänge im Zeeman-Bereich eine $10^{25}$ geringere Wahrscheinlichkeit für einen spontanen Übergang vorliegt. Diese Übergänge sind daher im Vergleich zu induzierter Emission hier vernachlässigbar.

\section{Aufbau und Durchführung}
Die Apparatur besteht im wesentlichen aus einer beheizten Dampfzelle und einer Spektrallampe mit einer optischen Schiene, auf der Filter und Linsen angebracht sind. Die Linsen werden so ausgerichtet, dass die Probe durchstrahlt wird und das Licht anschließend auf eine Photodiode gebündelt wird. Den genauen Aufbau kann man Abbildung \ref{pic_aufbau} entnehmen
\begin{figure}
\includegraphics[scale=1]{../pics/aufbau.jpg}
\caption{Die im Versuch benutzte Apparatur}
\label{pic_aufbau}
\end{figure}
Hochfrequenzfeld zum Pumpen und der von der Diode gemessene Strom werden dann an einen XY-Oszilloskop angeschlossen, wo Resonanzstellen sichtbar werden, wenn die richtige Frequenz getroffen wurde.\\

Vor der eigentlichen Messung muss jedoch zunächst das Erd





\section{Durchführung}
\label{sec_durch}

\section{Auswertung}
\subsection{Vertikalkomponente des Erdmagnetfelds}
\begin{table}[H]
 \begin{tabular}{l|c|c|c}
  Spule & $r$ in cm & $N$ & $I/U$ in A/1\\
  \hline
  Vertikal & 11,735 & 20 & 1\\
  Horizontal & 15,790 & 154 & 0,1\\
  Sweep & 16,39 & 11 & 0,3
 \end{tabular}
\caption{Kenndaten benutzter Helmholtzspulen}
\label{tab_SpulenHor}
\end{table}
Um dem Einfluss des vertikalen Magnetfelds entgegenzuwirken, wird eine Helmholtz-Spule in die entsprechende Richtung orientiert, sodass das vom 
Stromfluss $I$ induzierte Magnetfeld $B$ genau dem des Erdmagnetfelds entspricht. $\mu_0$ ist die Vakuumpermeabilität, $r$ der Radius und $N$ die Windungszahl
Bei einer Umdrehungszahl von $U$=1,99 V und einem Widerstand von $R$ = 1 $\Omega$ wird kein Einfluss des vertikalen Erdmagnetfelds mehr festgestellt. 
Die Feldstärke $B_{\text{vert,Erde}}$ lässt nach Biot-Savart und mit Tabelle \ref{tab_SpulenHor} berechnen mit
\begin{equation}
 B_{\text{vert,Erde}} = \mu_0 \frac{8}{\sqrt{125}}\frac{I N}{r} = 30,5 \, \mu\text{T}.
\end{equation}
\subsection{Horizontalkomponente des Erdmagnetfelds}
Wie in \ref{sec_durch} beschrieben, wird die Resonanzfrequenz der RF-Spule stetig erhöht und ebenfalls das Magnetfeld der Horizontalspulen dahingehend,
dass die Transmission der Dampfzelle stark abnimmt. In Tabelle \ref{tab_SpulenHor} sind die Kenndaten der Horizontalspulen aufgeführt, die zur Umrechnung
der in den Tabellen \ref{tab_nuB1} und \ref{tab_nuB2} gelisteten Messwerte in Magnetfelder nötig sind.


\begin{minipage}{0.5\textwidth}
\begin{table}[H]
\begin{tabular}{c|cc|c}
$\nu$ in kHz & U$_{\text{sweep}}$ & U$_{\text{hor}}$ & $B_{r,1}$ in mT\\
\hline
 20 &	1,75&	0,00&	0,011\\
100&	3,70&	0,00&	0,022\\
200&	1,05&	0,12&	0,038\\
300&	1,55&	0,15&	0,049\\
400&	3,82&	0,15&	0,063\\
500&	2,48&	0,24&	0,078\\
600&	2,13&	0,30&	0,092\\
700&	2,27&	0,35&	0,106\\
800&	2,38&	0,40&	0,120\\
900&	2,60&	0,45&	0,134\\
1000&	2,73&	0,50&	0,148 \\
\end{tabular}
\caption{$\nu$ und $B_r$ des 1. Isotops}
\label{tab_nuB1}
\end{table}
\end{minipage}
\begin{minipage}{0,5\textwidth}
\begin{table}[H]
\begin{tabular}{c|cc|c}
$\nu$ in kHz & U$_{\text{sweep}}$ & U$_{\text{hor}}$ & $B_{r,2}$ in mT\\
\hline
20	&2,00&	0,00&	0,012\\
100&	4,90&	0,00&	0,030\\
200&	3,65&	0,12&	0,054\\
300&	5,11&	0,15&	0,070\\
400&	4,25&	0,25&	0,091\\
500&	2,74&	0,37&	0,114\\
600&	2,59&	0,45&	0,134\\
700&	1,78&	0,55&	0,155\\
800&	1,05&	0,65&	0,177\\
900&	2,47&	0,70&	0,199\\
1000&	1,55&	0,80&	0,220 \\
\end{tabular}
\caption{$\nu$ und $B_r$ des 2. Isotops}
\label{tab_nuB2}
\end{table}
\end{minipage}

Der Zusammenhang zwischen $\nu$ und $B_r$ wird zweimal mit einem linearen Ansatz
\begin{equation*}
 \nu = a\cdot B + b \hspace{2cm} \text{und} \hspace{2cm} \nu = c\cdot B + d
\end{equation*}
getestet und die Koeffizienten $a,b,c$ und $d$ durch GNUplot bestimmt. In Abbildung \ref{pic_nuB} sind die Messwerte dargestellt, sowie die eben
genannten linearen Ansätze als Fitgeraden.
\begin{figure}[h]
\includegraphics[width=0.8\textwidth]{../pics/v21B-nu.pdf}
\caption{Zusammenhang zwischen Resonanzfrequenz und Magnetfeld}
\label{pic_nuB}
\end{figure}
Hieraus ergeben sich die folgenden zwei Gleichungen
\begin{align}
 \nu_1 =& 7159(1\pm0,67\%)\frac{\text{kHz}}{\text{mT}}\cdot B - 58,6(1\pm6,1\%) \text{kHz}
 \label{eq_nu1}
 \end{align}
 \begin{align}
 \nu_2 =& 4741(1\pm0,70\%)\frac{\text{kHz}}{\text{mT}}\cdot B - 39,6(1\pm11,1\%) \text{kHz}.
 \label{eq_nu2}
\end{align}
Mit diesen zwei Gleichungen ist es nun möglich die Horizontalkomponente des Erdmagnetfelds zu ermitteln. Sie ist genau das Magnetfeld $B$, welches
die Gleichung \eqref{eq_nu1} bzw. \eqref{eq_nu2} 0 werden lässt. Aus den zwei Werten wird anschließend der Mittelwert genommen
\begin{align}
 B_1 = 8,19(1\pm6,1\%) \mu T \hspace{0.2cm} \text{und} \hspace{0.2cm} B_2 = 8,35(1\pm11,1\%) \mu T,
\end{align}
was schließlich zu einer Horizontalkomponente führt von
\begin{equation}
 B_{\text{hor}} = 8,27(1\pm12,7\% )\mu T.
\end{equation}
\subsection{Landé-Faktoren des Atoms}
\label{sec_lande}
Neben der Horizontalkomponente des Erdmagnetfelds kann man aus den Gleichungen \eqref{eq_nu1} und \eqref{eq_nu2} ebenfalls die Landé-Faktoren des
Atoms $g_F$ nach Gleichung \eqref{eq_nuB.g_F} errechnen, wo der Proportionalitätsfaktor mit $a$ bzw. $c$ identifiziert wird
\begin{equation}
 g_{F,1} = 0,511(1 \pm 0,67\%) \hspace{1cm}\text{und}\hspace{1cm}g_{F,2} = 0,339(1\pm 0,7\%) 
\end{equation}
Desweiteren lassen sich aus der Elektronenkonfiguration von Rubidium \cite{EKonf} die Drehimpulse bestimmen, sowie der Landé-Faktor der Elektronenhülle $g_J$.
Die Drehimpulse sind hierbei
\begin{equation}
 L = 0 , \qquad S=\frac12 , \qquad J = L+S = \frac12, \qquad F = I+J= I + \frac12,
\end{equation}
was nach \eqref{eq_gj} zu einem Faktor führt zu
\begin{equation}
 g_J = 2,0023. 
\end{equation}

\subsection{Kernspin $I$ der Rubidiumisotope}
Mit den Ergebnissen aus \ref{sec_lande} lassen sich nun nach \eqref{eq_gF(I)} die Kernspins der auftretenden Rubidiumisotope errechnen. Die etwas
längliche Formel ergibt umgestellt nach dem Kernspin
\begin{equation}
 I_k = \frac{1}{4\frac{g_{F,k}}{g_J}} \left[\left(1-4\frac{g_{F,k}}{g_J}\right) + \sqrt{\left(-1+4\frac{g_{F,k}}{g_J}\right)^2-12\frac{g_{F,k}}{g_J}\left(\frac{g_{F,k}}{g_J}-1\right)}\right]
\end{equation}
und führt zu
\begin{equation}
 I_1 = 1,459 \approx \frac32 \hspace{3cm} I_2=	2,459 \approx \frac52.
\end{equation}

\subsection{Isotopenverhältnis von $^{85}$Rb und $^{87}$Rb}
Durch das Auftauchen von zwei Resonanzfrequenzen bei denen die Transparenz der Probe einbricht, wird davon ausgegangen, dass es sich um zwei verschiedene
Isotope innerhalb der Probe handelt. Das Verhältnis ihres Vorkommens $N_i$ hängt mit dem Verhältnis der Transparenzaufhebung $A_i$ direkt zusammen
Die Bestimmung der Amplituden geschieht durch Ablesen am Oszilloskop in Abbildung \ref{pic_verh}
\begin{figure}[h]
\includegraphics[width=0.8\textwidth]{../pics/v21Verh.jpg}
\caption{Typische Aufnahme am Oszilloskop, hier bei $\nu$ = 100 kHz}
\label{pic_verh}
\end{figure}\begin{equation}
 \frac{N_1}{N_2} = \frac{A_1}{A_2} = \frac{5,5}{2,5}= 2,2.
\end{equation}

\section{Diskussion}
\subsection{Erdmagnetfeld}
Die ermittelten Werte für die Vertikal- und Horizontalkomponente des Erdmagnetfelds sind anfolgend mit den Literaturwerten \cite{ErdB} verglichen. Die erheblichen
Fehler sind wohl auf ein ungenau Ausrichtung der Apparatur in Nord-Süd-Richtung zurückzuführen
\begin{equation}
 \frac{B_{\text{vert}}}{B_{\text{vert,Lit}}} = 70\% \hspace{3cm} \frac{B_{\text{hor}}}{B_{\text{hor,Lit}}} = 41\%
\end{equation}
\subsection{Eigenschaften von Rubidium}
Die erittelten Werte für die Landé-Faktoren $g_F$ haben zu den zwei Kernspins $I_1$ und $I_2$ geführt und werden anhand einer Nuklidkarte \cite{Spin} Rubidiumisotope
zugewiesen
\begin{equation}
 I_1 \approx \frac32 \rightarrow\, ^{87}\text{Rb} \hspace{2cm} I_2 \approx \frac52 \rightarrow\, ^{85}\text{Rb}.
\end{equation}
Das Verhältnis der Rb-Isotope \cite{Spin} wird mit dem Verhältnis der Amplituden verglichen, was zu folgender Übereinstimmung führt
\begin{equation}
 \frac{A_1}{A_2}\, /\, \frac{N_{^{85}\text{Rb}}}{N_{^{87}\text{Rb}}} = 2,2 \,/\, \frac{72,17\%}{27,83\%} =85\% .
\end{equation}



\begin{thebibliography}{xxxxxxxxxxx}
 \bibitem[PSE]{EKonf}Periodensystem der Elemente\\ \href{http://www.periodensystem.info/elemente/rubidium/}{periodensystem.info/elemente/rubidium/}
 \bibitem[Chemie.de]{ErdB}Form und Stärke des Erdmagnetfelds\\ \href{http://www.chemie.de/lexikon/Erdmagnetfeld.html#Form_und_St.C3.A4rke_des_Erdmagnetfeldes}{chemie.de/lexikon/Erdmagnetfeld}
 \bibitem[KAERI]{Spin}Nuklidkarte des \textit{Korea Atomic Energy Research Institute}\\ \href{http://atom.kaeri.re.kr/}{atom.kaeri.re.kr/}
\end{thebibliography}



\parskip 340pt
\Large{Literatur}\\\\

% ========================================
%	Literaturverzeichnis
% ========================================

%\bibliographystyle{plainnat}			% Bibliographie-Style auswählen
%\bibliography{BIBDATEI}			% Literaturverzeichnis

% ========================================
%	Das Dokument endent
% ========================================

\end{document}
